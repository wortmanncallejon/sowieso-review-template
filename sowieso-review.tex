\documentclass[11pt,a4paper]{article}
\usepackage{graphicx}
\usepackage[utf8]{inputenc}
\usepackage{epsfig,graphicx}
\usepackage{hyperref}
\usepackage{enumitem}
\usepackage{amsmath}
\usepackage{float}
\usepackage{siunitx}
\usepackage[left=2cm,right=2cm,top=2cm,bottom=2cm,includehead]{geometry}
\usepackage{amssymb}
\usepackage{fancyhdr}
\usepackage{multicol}
\usepackage{libertine}
\usepackage{libertinust1math}
\usepackage[T1]{fontenc}
\usepackage[style=mla]{biblatex}
\addbibresource{refs.bib}


\addtolength{\topmargin}{-46.75916pt}
\setlength{\headheight}{50pt}
\begin{document}
\pagestyle{fancy}

\fancyhead[L]{\includegraphics[height=1.5cm]{logo.png}}
\fancyhead[R]{\textbf{Reviewer:} \href{mailto:review@sowieso-journal.de}{Reviewer Name} \\  \textbf{Submission-ID:} AXXXXX-XXXA}
%\fancyhead[R]{\textbf{Reviewer:} \texttt{Anonymised} \\  \textbf{Submission-ID:} AXXXXX-XXXA}
\fancyfoot[C]{\large \thepage}
 
\begin{center}
\textbf{\Large Paper Title}\\
{\large Subtitle}
\end{center}

\begin{multicols}{2}
\section{Summary of the Paper}
The author addresses a key question in political science by investigating [research question], contributing to the literature in two primary ways: first, by introducing a novel theoretical concept [concept], and second, by constructing an original dataset consisting of [data description]. The analysis employs regression techniques to examine [main relationship], finding evidence that supports the primary hypothesis and mixed results for a secondary hypothesis regarding [secondary relationship]. The paper concludes with implications for policy and theory, particularly in relation to [broader issue or field].

\section{Formalities}
The manuscript is written in [adjec] language and adheres to formal academic standards, although minor issues with formatting and citation consistency are noted. The author uses an author-year citation style and maintains a logical structure through effective use of headings and subheadings. Some issues with spelling consistency (e.g., American vs. British English) and word/character limit compliance were noted, with the text exceeding the limit by approximately $X\%$.

\section{Strengths of the Paper}
The paper demonstrates several key strengths:
\begin{itemize}
    \item Innovative use of [quantitative/qualitative] methods, including [specific methods or tools].
    \item A well-constructed dataset that enhances the empirical foundation of the paper.
    \item A bold approach to theory-building, which connects existing literature with new insights on [research topic].
    \item Clear and effective use of visualizations to support the data analysis.
\end{itemize}
These elements contribute to the paper's potential impact on the field.

\section{Weaknesses of the Paper}
Despite the strengths, several issues need addressing:
\begin{itemize}
    \item The theoretical framework could be better integrated with broader political science paradigms. The reliance on specific authors and models (e.g., [author/model]) limits the theoretical depth.
    \item The empirical analysis could benefit from additional clarity regarding model selection, especially in terms of statistical assumptions and interpretation of results.
    \item The presentation of results, especially regarding coefficients and interaction effects, lacks transparency, making it difficult to fully assess the robustness of the findings.
    \item Further discussion of limitations, particularly concerning causality, would strengthen the paper's validity.
\end{itemize}
Addressing these weaknesses would significantly enhance the paper's overall quality.


\section{Recommendation to the Editors}
The paper offers valuable contributions to the field of political science, particularly in its innovative methodological approach and theoretical exploration. However, due to the concerns outlined above, I recommend a \textbf{revise and resubmit} decision. With revisions focused on theoretical integration, empirical transparency, and methodological rigor, this paper could make a strong contribution to the literature.

\end{multicols}

\newpage

\printbibliography[title=Bibliography]


\end{document}

